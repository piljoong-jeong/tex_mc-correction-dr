% \documentclass[acmtog,anonymous,review]{acmart}
% \acmSubmissionID{1234}
\documentclass[acmtog,review]{acmart}

\usepackage{booktabs} % For formal tables
\usepackage{xcolor}
\definecolor{steelblue}{rgb}{0.27, 0.51, 0.71}
\newcommand{\PJ}[1]{\textcolor{steelblue}{\bfseries{PJ: {#1}}}}
\usepackage{amsmath}
% \usepackage{amssymb} % symbol already defined
% \usepackage{newtxmath}
% \usepackage{preamble-math}

% \usepackage{newunicodechar}
% \newunicodechar{⊕}{reg_op}

\usepackage[T1]{fontenc}
\usepackage{fixltx2e}

% TOG prefers author-name bib system with square brackets
\citestyle{acmauthoryear}
%\setcitestyle{nosort,square} % nosort to allow for manual chronological ordering



\usepackage[ruled]{algorithm2e} % For algorithms
\renewcommand{\algorithmcfname}{ALGORITHM}
\SetAlFnt{\small}
\SetAlCapFnt{\small}
\SetAlCapNameFnt{\small}
\SetAlCapHSkip{0pt}

% Metadata Information
\acmJournal{TOG}
%\acmVolume{38}
%\acmNumber{4}
%\acmArticle{39}
%\acmYear{2019}
%\acmMonth{7}

% Copyright
%\setcopyright{acmcopyright}
%\setcopyright{acmlicensed}
%\setcopyright{rightsretained}
%\setcopyright{usgov}
%\setcopyright{usgovmixed}
%\setcopyright{cagov}
%\setcopyright{cagovmixed}

% DOI
%\acmDOI{0000001.0000001_2}

% Paper history
%\received{February 2007}
%\received{March 2009}
%\received[final version]{June 2009}
%\received[accepted]{July 2009}



% Document starts
\begin{document}

% Title portion
\title{Single-view TSDF Mesh Noise Reduction under Virtual Light using Differentiable Rendering}

% DO NOT ENTER AUTHOR INFORMATION FOR ANONYMOUS TECHNICAL PAPER SUBMISSIONS TO SIGGRAPH 2019!
\author{PilJoong Jeong}
\orcid{0000-0001-7579-2047}
\affiliation{
 \institution{Gwangju Institute of Science and Technology}
 \streetaddress{123 Cheomdangwagi-ro}
 \city{Buk-gu}
 \state{Gwangju}
 \postcode{61005}
 \country{Republic of Korea}}
\email{piljoong.jeong@gm.gist.ac.kr}

%\renewcommand\shortauthors{Jeong, P. et al}

\begin{teaserfigure}
    \includegraphics[width=\textwidth]{figures/0_teaser_overview_rev3.png}
    \caption{Overview of our method. (From top left) We saturate noisy vertices by rendering input TSDF mesh with virtually placed light source. We extract common-shared geometric clue between rendered input mesh and target color image. Differentiable renderer iteratively minimizes loss, which is the difference between clues from rendered input mesh and target Ground Truth color image. Orange, Yellow, and Blue inset shows difference between input mesh and optimized mesh. Our method successfully reduces noise in mesh vertices. Furthermore, our provided video shows input mesh that is being optimized (right-side of the video) as iteration continues, as well as the visualization of loss (left-side of the video) that is being minimized: \url{https://drive.google.com/file/d/10F_I89m5O-RWOIxocYoxG2QOkJc7YWlF/view?usp=sharing}}
    \label{fig:one}
\end{teaserfigure}

\begin{abstract}
Thanks to consistent evolution of SLAM (Simultaneous Localization and Mapping) and its related technologies, we can reconstruct geometric properties of where we are currently observing in real-time. Due to the limitation of current depth sensing hardware, however, we are generally able to obtain geometric features corrupted by noise. Color image is perceived as geometrically noise-free in terms of human vision-perception system, but to our best knowledge, encoding the information from Ground Truth for differentiable rendering under single view constraint is not discussed yet. In this report, we propose a bridge between the geometric information generated from color image and rendered mesh, so that differentiable renderer can optimize input i.e., noisy vertex position without any depth supervision. The key insight is that we can highlight noisy vertices by rendering mesh with virtually placed light sources. We compare our result with one of the state-of-the-art differentiable rendering method[1], and show our method outperforms previous method which requires prior depth information (silhouette image).
\end{abstract}


%
% The code below should be generated by the tool at
% http://dl.acm.org/ccs.cfm
% Please copy and paste the code instead of the example below.
%
\begin{CCSXML}
<ccs2012>
    <concept>
        <concept_id>10010147.10010371.10010396</concept_id>
        <concept_desc>Computing methodologies~Shape modeling</concept_desc>
        <concept_significance>500</concept_significance>
        </concept>
    <concept>
        <concept_id>10010147.10010371.10010387.10010392</concept_id>
        <concept_desc>Computing methodologies~Mixed / augmented reality</concept_desc>
        <concept_significance>500</concept_significance>
        </concept>
    </ccs2012>
\end{CCSXML}

\ccsdesc[500]{Computing methodologies~Shape modeling}
\ccsdesc[500]{Computing methodologies~Mixed / augmented reality}

%
% End generated code
%


\keywords{Differentiable rendering, Mesh denoising}



\maketitle

% \input{samplebody-journals}
\section{Introduction}

To satisfy increasing demand of 
Augmented Reality (AR) technology, we required to obtain geometric information of observed real scene as precise as possible. 
As precision of geometric detail is increased, we can augment virtual objects with more consistency from computer graphics field (rendering) perspective, and we are able to estimate camera poses from perfect correspondences from computer vision field (SLAM) perspective.

Capturing detailed geometry is still remain as technical challenge due to hardware/computational limitation of SLAM. 
\PJ{TODO: actually, this is not true. Fine voxels can represent geometric detail but easily hindered by noise, whereas coarse mesh has resilience against noise but fails to represent where geometric detail is required e.g., edges}
It is obvious that consumer-level depth sensing technology still has noisy observation. 
We can cope this by divide a real scene with detailed (fine) voxels as precise as possible when generating TSDF mesh. 
In this way, however, unnecessarily large number of triangles are generated even in the case of reconstructing simple geometry (e.g., plane). 
This directly affects to rendering performance, as rendering requires primitive traversal in order to appropriately propagates lighting information for every single frame. 
Due to those limitations, we are compromised to use TSDF meshes from coarse voxels, which have geometric incorrectness including noise. 
Nevertheless, there is strong demand to perfect geometry captured from SLAM sequences. 

Recent advances of Differentiable Rendering make it possible to optimize current input parameters by observing a set of given Ground Truth images. 
This seems promising to SLAM, as they naturally capture Ground Truth color images whereas generating input TSDF mesh corrupted by noisy measurements. 
However, it is unclear that how to interpret (perceptually encoded) geometric clues within color images, reflect those information into actual geometry to minimize its imperfection. 
Moreover, there is some limitations hinder directly applying previous differentiable rendering techniques into SLAM dataset. 
We explained such difficulties in \ref{fig:difference_simple_mesh_and_tsdf_mesh}.

Our main contribution is bridging the gap between perceptional noise-free geometric features from $\mathcal{C}$ and noisy geometric feature in $\mathcal{M}$. 
To exploit this, we borrow a novel concept of image denoising using flashlight. Images taken with flashlight can hold additional features which are hard to detect from general geometric feature (e.g., depth, normal etc). 
For example, in \cite{eisemann2004flash} \cite{petschnigg2004digital} flashy photography is used to enhance images taken from scene which has insufficient lighting condition. 
\cite{moon2013robust} is pioneering work that adopt image enhancement using flashlight on photorealistic rendering domain, by casting virtual flashlights to capture a scene’s reflective / refractive features, which are not stored in traditional G-buffers. 
Based on these approaches, we saturate noisy vertices by casting virtual light, which are never detected when rendered with mesh’s albedo only. 
Please refer the leftmost image (rendered without light) and its connected image (rendered with virtual light) in \ref{fig:teaser} for details. 
We demonstrate our results, compare with result from previous method, and show that our method outperforms previous result.
\section{Previous Works}

\paragraph{TSDF Mesh Noise Reduction in SLAM}
Starting from pioneering work \cite{curless1996volumetric}, 
estimating true depth from noisy measurements 
has been one of the major challenges in SLAM. 
Basically, \cite{curless1996volumetric} first introduced the definition of ‘fusion’, 
which is accumulating noisy world positions into a voxel. 
This acts as similar with spatial running average filter, 
therefore it is known that accumulating frames 
that captures same region can reduce noise incrementally. 
However, this takes a lot of time to converge depth values 
to a true mean, which interferes practical AR application experience. 
Therefore, various depth noise reduction techniques are applied to SLAM systems, 
including simple bilateral filter \cite{newcombe2011kinectfusion},  merging depth information of neighboring frames either offline \cite{choi2015robust}\cite{zhou2018open3d}, or online \cite{cao2018real}\cite{yang2020noise}. 
All mentioned previous works require multiple depth frames 
in order to generate reliable mesh which takes a long time, 
and they did not consider color images, which holds perceptive geometric information. 
Our method, in contrast, is able to infer geometric clue in color image, 
thus able to de-noise input mesh without multiple frames.

\paragraph{Differentiable Rendering}
Differentiable rendering is the technique that 
optimizing input parameters by observing a set of target images 
via iteratively render a scene with current state of input parameters (i.e., forward pass), 
as well as propagate gradients along with computation graph (i.e., backward pass) built at forward pass. 
Due to the fact that it does not require any prior knowledge (e.g., pre-trained model trained from external dataset) 
other than target images, this seems an off-the-shelf optimizer for SLAM 
since we naturally capture target image during scanning process, 
whereas fused mesh is contaminated with noise due to its nature limitations. 
However, there are ambiguities that have to be considered 
in order to adapt differentiable rendering to optimize SLAM problems. 
\ref{fig:difference_simple_mesh_and_tsdf_mesh} describes the different setup between simple mesh optimization 
and indoor TSDF mesh optimization. 
Our method bypasses those ambiguities via inferring geometric clue 
without necessity of silhouette information, as well as multiple target observations.
\begin{figure}
    \includegraphics[width=\columnwidth]{figures/2_prev_difference_simple_mesh_and_tsdf_mesh.png}
    \caption{Different setup between optimizing simple mesh and TSDF mesh from SLAM. Previous works aimed to optimize input geometry to set of target images captured with different camera view either synthetically\cite{ravi2020accelerating}\cite{jatavallabhula2019kaolin}\cite{laine2020modular}\cite{nimier2019mitsuba} or by taking calibrated real photographs\cite{nimier2019mitsuba}. However, in the case of TSDF mesh from SLAM it is ambiguous since (1) input is not separated with its backgrounds, hence silhouette image is hard to generate (2) we cannot generate synthetic target views (3) it is hard to sample real images from SLAM sequences which captures a region that input mesh represents, since the labeled camera pose paired with target image is estimated value. It is well-known that camera poses from SLAM is inaccurate. (a) It is straightforward to generate synthetic target images if the target image can be rendered. (b) Unlike (a), it is much hard to augment target images in order to optimize indoor TSDF mesh.}
    \label{fig:difference_simple_mesh_and_tsdf_mesh}
\end{figure}
\section{Methods}

We propose an optimization procedure that incrementally minimizes mesh vertex noises using differentiable rendering. 
Suppose we have vertices $V=\{V_i\in\mathbb{R}^3\}=\{V_0...V_n\}$ within input mesh $\mathcal{M}$ which is fused from input color $\mathcal{C}$ and depth $\mathcal{D}$ with intrinsic $K$ such that $\mathcal{M}=K^{-1}\left(\mathcal{C}\oplus \mathcal{D}\right)$. 
Here, $\oplus$ denotes image registration operator.
We define deformed vertices $V_d$, which has same shape as $V$ initialized with zero values i.e., $V_d=\{\mathbf{0}_0,...,\mathbf{0}_n\}$. 
For every iteration, $V_d$ is optimized so that resulting vertices $V_o=V+V_d$ approximate noise-free geometry, so that its rendering $C$ looks perceptually similar with $\mathcal{C}$. 
The main problem is how to find common geometric representations between $\mathcal{C}$ and $C$ to figure out which region is to be optimized (i.e., noisy). 

\subsection{Exploting assumptions underlying existing SLAM datasets}
We found two common aspects within $\mathcal{C}$ taken from most of SLAM dataset, which acts important role to consider $\mathcal{C}$ as geometric clue.

\noindent \textbf{Lambertian-dominant}. 
This naturally satiesfies since this is an assumption for vast majority of SLAM dataset, unless it is used for vaildating SLAM algorithm under non-Lambertian environment e.g., \cite{whelan2018reconstructing}. 
In order to track accurate camera pose during SLAM, finding reliable feature correspondences are necessary.
Generally, a point shown in two images that shares overlapped region $\mathcal{C}_0$, $\mathcal{C}_1$ is chosen, which satisfies photometric consistency e.g., $E_{photo}=\arg\min_x \mathcal{C}_0\left(x+\overrightarrow{\mathrm{u}}\right)-\mathcal{C}_1\left(x\right)$. \cite{szeliski2010computer}
This is because it has been empirically considered as 'static anchor' on a given scene; a point on a surface never deforms during capturing stage, and its value is direction-invariant so that the point has consistent pixel value wherever an image is taken from. 

Interpreting photometric consistency in computer graphics field is trivial; this indicates that a point is lie on a Lambertian surface, such that a bi-directional reflectance $f(x, w_i, w_o)=c$, where $c\le 1$ is a constant.


\noindent \textbf{No strong radiance changes within a plane}. 
We found that there seldom have strong radiance changes within a plane among indoor SLAM datasets. 
Here, a plane can be a structural information of given scene (e.g., wall, floor, ceiling), or upper part of an object (e.g., desk, box).
Although there may exists a case that a plane have strong radiance changes (e.g., chessboard pattern on floor), we decided to ignore them. 
In other words, any point within a plane have similar pixel value with neighbors lie on same plane. 
Note that our proposed method works even though this assumption does not met, however we decide to made it for the sake of ease of explanation.


\subsection{Re-vising: How Geometry Affects to Color Image}
In this section, we formally describe our intuition that enables input color image treated as noise-free geometric clue.

Let us consider $\mathcal{C}$ as a result from perfect renderer which is capable of tracing full light transport without any noise or outlier, given unknown lighting conditions and noise-free geometry. 
Based on Light Transport Equation(LTE) representation with respect to path integral form \cite{veach1998robust}, we represent $\mathcal{C}$ as a set of solution of LTE for each pixel:

\begin{align}
    \mathcal{C} & = \bigcup_i^W \bigcup_j^H L_{i,j}\left(K^{-1}\cdot x\rightarrow p_0\right) \nonumber \\
    & = \bigcup_i^W \bigcup_j^H L_{i,j}\left(p_1\rightarrow p_0\right),
    \label{eqn:color_image_rendering_equation}
\end{align}
where $L_{i,j}$ is total radiance at a pixel, and $p_1=K^{-1}\cdot x$ is a hitpoint corresponds to $L_{i,j}$.

To intuitively exploit relationship between $\mathcal{C}$ and noise-free hitpoint $x_{i,j}$, we select arbitrary pixel and its total radiance in (Eqn. \ref{eqn:color_image_rendering_equation}) and expand radiance sum over path segments $\bar{\mathrm{p}}_n=\mathrm{p}_0\mathrm{p}_1...\mathrm{p}_n$ with $n+1$ vertices:
\begin{align}
    L_{i,j}\left(p_1\rightarrow p_0\right)& = \mathit{P}\left(\bar{\mathrm{p}}_1\right)+\mathit{P}\left(\bar{\mathrm{p}}_2\right)+\sum_{n=3}^\infty \mathit{P}\left(\bar{\mathrm{p}}_n\right), 
    \label{LTE_path_integral}
\end{align}

Let us consider direct lighting term in (Eqn. \ref{LTE_path_integral}) first:

\begin{align}
    \mathit{P}\left(\bar{\mathrm{p}}_2\right) & = \int_A && f\left(\mathrm{p}_2\rightarrow \mathrm{p}_1 \rightarrow \mathrm{p}_0\right)\cdot L_{e(i,j)}\left(\mathrm{p}_2\rightarrow \mathrm{p}_1\right) \cdot G\left(\mathrm{p}_1 \leftrightarrow \mathrm{p}_2\right) \nonumber \\
    & = \int_A && \Bigg[ f\left(\mathrm{p}_2\rightarrow \mathrm{p}_1 \rightarrow \mathrm{p}_0\right)\cdot L_{e(i,j)}(\mathrm{p}_2\rightarrow \mathrm{p}_1) \nonumber \\ 
    & && \cdot V(\mathrm{p}_1 \leftrightarrow \mathrm{p}_2) \cdot \frac{|\cos(\theta_1)| |\cos(\theta_2)|}{||\mathrm{p}_1-\mathrm{p}_2||^2}\Bigg]
    \label{LTE_path_integral_direct_lighting}
\end{align}

Note that we omitted path differential $dA(\mathrm{p}_2)$ for brevity.

Since we assumed Lambertian surface and emitted radiance term is free up to any geometric transformation, the term that affects to direct lighting value in (Eqn. \ref{LTE_path_integral_direct_lighting}) is \textbf{geometry term}: visilbility, incident angles, and distance between hitpoints. \PJ{TODO: parameterize hitpoint as a sum of barycentric coords of vertex positions within a face.}

We will drop any consideration of remaining terms in (Eqn. \ref{LTE_path_integral}). We do not consider emitted radiance i.e., albedo term $\mathit{P}\left(\bar{\mathrm{p}}_1\right)$ since this represents pure physical quantity of a geometry have, thus itself is consistent up to any geometric displacement (e.g., additive noise on vertices).
\PJ{Does it sound clear?}
We additionally treat indirect lighting term $\sum_{n=3}^\infty \mathit{P}(\bar{\mathrm{p}}_n)$ as it is; we will consider that indirect lighting cannot make a pixel value strongly deviated with neighbors where is originally have similar value within neighbors when only direct lighting is applied. 
This seems feasible because (1) indirect lighting term uses rendering equation, which is identical to direct lighting term except to hitpoints (2) its throughput is far smaller than direct lighting term, under Lambertian assumption (3) it is applied globally; to make a pixel saturated against neighbors indirect lighting have to be applied to only that pixel (4) in a photorealistic rendering domain, multiple bounces among diffuse material causes significant noise, hence often ignored\cite{moon2013robust}.

Therefore, we conclude that radiance at a pixel $L_{i,j}(p_1 \rightarrow p_2)$ is mostly affected by \textbf{geometry term} in $\mathit{P}(\bar{\mathrm{p}}_2)$ i.e., 
\begin{align}
    L_{i,j}(p_1 \rightarrow p_2) \sim \mathit{P}(\bar{\mathrm{p}}_2) & \propto G(\mathrm{p}_1 \leftrightarrow \mathrm{p}_2) \nonumber \\
    & =V(\mathrm{p}_1 \leftrightarrow \mathrm{p}_2) \cdot \frac{|\cos(\theta_1)||\cos(\theta_2)|}{||\mathrm{p}_1-\mathrm{p}_2||^2}
    \label{eqn:relationship_between_radiance_and_geometry}
\end{align}

We now examine direct lighting term in rendering $C$, where noisy geometry $\mathcal{M}$ is used.

\begin{align}
    \mathit{P}\left(\bar{\mathrm{p}}'_2\right) = c\int_A L_{e(i,j)}(\mathrm{p}'_2\rightarrow \mathrm{p}'_1) \cdot V(\mathrm{p}'_1 \leftrightarrow \mathrm{p}'_2) \cdot \frac{\left|\cos(\theta_1)'\right|\left|\cos(\theta_2)'\right|}{\left|\left|\mathrm{p}'_1-\mathrm{p}'_2\right|\right|^2} \nonumber
    \label{LTE_path_integral_direct_lighting_SLAM}
\end{align}

As $\mathcal{M}$ is composed with noisy vertices $V$, \textbf{geometry term} is highly likely to have different value against those from $\mathcal{C}$. 
Thus, there may exist different radiance values between $\mathcal{C}$ and $C$ though evaluated at an identical pixel position.

This naturally concludes that a loss minimizing radiance difference equals to matching \textbf{geometry term} in $C$ similar with $\mathcal{C}$.

However, simply minimizing color difference i.e., $||\mathcal{C}-C||^2$ does not work, as the texture of $\mathcal{M} \sim \mathcal{C}$ already contains real light information, furthermore we will cast a virtual light to saturate noisy vertices. 
Therefore, $C$ have far different color tone compared to $\mathcal{C}$ in a global manner. 
The main challenge is to detect geometric differences between images in a local manner, regardless of any radiance tone changes.

\subsection{Detecting Geometric Information given Color Image and Its Rendering}
In this section, we describe our method to detect geometric information in both $\mathcal{C}$ and $C$, based on our intuition in (Eqn. \ref{eqn:relationship_between_radiance_and_geometry}).

We showed that changes of value made in a pixel is proportional to geometric transformation of a corresponding hitpoint.
Consider a pair of hitpoint and corresponding pixel, together with neighboring pixels and corresponding hitpoints lie on same plane.
If pixel values are evaluated in $\mathcal{C}$, they will have similar radiance values according to assumption. 
This can be expressed as:
\begin{equation}
    |L(K^{-1}\cdot x_{i,j})-L(K^{-1}\cdot x_{i+\delta_i, j+\delta_j})| < \epsilon, 
    \label{eqn:pixel_difference_near_epsilon}
\end{equation}
where $\delta_i \neq 0, \delta_j \neq 0$ is pixel coordinate displacement indicating neighbor of a pixel.

Based on our intuition, we decided to apply image gradient kernel to detect geometric displacement. 
This is natural as image gradient saturates where pixel values are discontinuous along with a pixel and its neighbors.
From our assumption and intuition, this directly indicates that saturated region equals to the position where is geometrically discontinuous.
If we apply image gradient kernel $G(\cdot)$ to $\mathcal{C}$, $G(\mathcal{C})$ have zero values where satisfies (Eqn. \ref{eqn:pixel_difference_near_epsilon}), and vice versa.

We simply detect noise-free surfaces within $\mathcal{C}$ by applying Scharr gradient kernel. 
In order to ensure robustness on strong texture changes, one may adapt gradient from $\mathcal{D}$; note that for this report we only experimented with gradients from $\mathcal{C}$. 
Specifically, Scharr kernel for each axis over an image is defined as
\begin{equation}
    \label{eqn:01}
    G_x=\begin{pmatrix}
        -3 & 0 & 3\\
        -10 & 0 & 10\\
        -3 & 0 & 3
    \end{pmatrix}, 
    G_y=\begin{pmatrix}
        -3 & -10 & -3\\
        0 & 0 & 0\\
        3 & 10 & 3
    \end{pmatrix}
    ,
\end{equation}
and our detected geometric changes over Ground Truth color image $\widetilde{G}_\mathcal{C}$ is a Scharr gradient of $I_\mathcal{C}$ i.e., the intensity image from $\mathcal{C}$. 
Since resulting gradient on some pixels have negative value of identical magnitude to positive values of neighboring pixel, we take its absolute value
\begin{equation}
    \label{eqn:02}
    \widetilde{G}_\mathcal{C}=\frac{1}{2}\left(|G_x\left(I_\mathcal{C}\right)|+|G_y\left(I_\mathcal{C}\right)|\right)
\end{equation}
In order to detect noisy vertices on $C$ regardless of shading method, we propose Lightweight map to determine which vertex has stronger noise compare to $\widetilde{G}_\mathcal{C}$. 
Lightweight map $I\textsubscript{lw}$ is an image taken from identical camera setup to $\mathcal{C}$, but holds how much a pixel corresponds to a hitpoint is affected by virtual light at a shading stage.
$I\textsubscript{lw}$ is defined as
\begin{equation}
    I\textsubscript{lw}=\{I\textsubscript{lw,i}\in\mathbb{R}\textsuperscript{\textit{W}*\textit{H}}\}, I\textsubscript{lw,i}=\frac{\left(x_i-p_0\right)\cdot n_i}{d\left(x_i,p_0\right)+\epsilon}, 
\end{equation}
where $x_i$, $n_i$ is world position and normal of hitpoint with pixel index \textit{i}, respectively. 
$p_0$ is position of virtual light, \PJ{TODO. double check.} and $d\left(x_i, p_0\right)$ is Euclidean distance between hitpoint and light position. \PJ{TODO. double check.}
We obtain changing amount of each lightweight value $\widetilde{G}\textsubscript{lw}$ by applying Scharr kernel over $\textit{I}\textsubscript{lw}$, similar with $\widetilde{G}_\mathcal{C}$.
\begin{equation}
    \widetilde{G}_\textsubscript{lw}=\frac{1}{2}\left(|G_x\left(I\textsubscript{lw}\right)|+|G_y\left(I\textsubscript{lw}\right)|\right)
\end{equation}
We found that using geometric normal to calculate lightweight map saturate pixels around noisy vertices more than shading normal, as shading normal smooths normal of each hitpoint using neighboring vertex normal and its barycentric coordinates. 
In detail, pixels within same face have similar lightweight values since they are both geometrically close to each other, and they share same normal. 
Pixels that are geometrically close, but within different faces are highly likely to have similar values if two faces have near identical normal values. 
This is the case when two faces are considered as ‘flat’ to each other, meaning that shared vertices have no noise. 
As the vertex have bigger noise, the gap or normal between sharing two faces also gets bigger. 
This brings pixels in $G\textsubscript{lw}$ have large value where there is significant normal difference, meaning that the region has noisy vertex. 
Note that larger $G\textsubscript{lw}$ at a pixel means that the pixel has higher noise, therefore differentiable renderer can optimize the region more aggressively. 
This is illustrated in Figure 3.
We observed that the intensity image of rendered scene with virtual light $I_C$ serves similar role with $I\textsubscript{lw}$ as they properly reflect strong gradients around deviating normal.
Therefore, we note that for all results we used $I_C$ instead of $I\textsubscript{lw}$.
\begin{figure*}
    \centering
    \includegraphics[width=\textwidth]{figures/3_method_relationship_gradient_lightweight_and_noise_full_tone_changed.png}
    \caption{Relationship between $G\textsubscript{lw}$ and actual noise of vertex $\textcolor{Orange}{V}$. We manually selected three pixels, where each have different magnitude of $G\textsubscript{lw}$. We also illustrated 1D example of the relationship. \textbf{\textcolor
    {Gray}{Gray}} and \textbf{\textcolor{Purple}{Purple}} lines and arrows indicate each face and its geometric normal. \textbf{\textcolor{LimeGreen}{(a)}} $\textcolor{Orange}{V}$ is considered as noise-free since $G\textsubscript{lw}$ is evaluated as zero, meaning adjacent faces have identical normal value. \textbf{\textcolor{Dandelion}{(b)}} $G\textsubscript{lw}$ is increased as two faces have inconsistent normal. \textbf{\textcolor{Red}{(c)}} as $G\textsubscript{lw}$ gets bigger, $\textcolor{Orange}{V}$ is considered as highly-noisy vertex. From these examples, we can say that $G\textsubscript{lw}$ precisely indicates where noisy pixels exist.}
    \label{fig:relationship_gradient_lightweight_and_noise}
\end{figure*}

\subsection{Optimization using Differentiable Rendering}
We define lightweight loss to minimize the difference between noise-free geometric information $\widetilde{G}_\mathcal{C}$ and noise-detected rendered image $\widetilde{G}_C$. We found that there is gradient value range inconsistency between $\widetilde{G}_\mathcal{C}$ and $\widetilde{G}_C$, as they are derived from different type of image i.e., color and geometry, respectively. We applied hyperbolic tangent kernel to each gradient image to ensure that both images have normalized value range, and we observed that this helped optimizer to find optimal without failure. Finally, we replace our target GT image from $\mathcal{C}$ to $\mathcal{C}\oplus\mathcal{D}$, as $\mathcal{M}$ follows holes where pixels in $\mathcal{D}$ have zero value. Our final geometric gradients are:
\begin{equation}
    G_\mathcal{C}=\tanh\left(\widetilde{G}_{\mathcal{C}\oplus\mathcal{D}}\right),
    G_{lw}=\tanh\left(\widetilde{G}_{lw}\right), 
\end{equation}
where $\tanh\left(x\right)=\frac{e^x-e^{-x}}{e^x+e^{-x}}$. Our optimizer minimizes lightweight loss representing geometric difference, while penalizing vertices not to evolve too far from its initial position:
\begin{gather}
    \mathcal{L}=w_{lw}\cdot L_{lw}+w_{pos}\cdot L_{pos}, \\
    L_{lw}=\left|\left|G_\mathcal{C}-G_C\right|\right|^2_2, \nonumber \\
    L_{pos}=\left|\left|V-\left(V_o\right)\right|\right|^2_2=\left|\left|V_d\right|\right|^2_2 \nonumber
\end{gather}
For all results, we used $w_{lw}=0.01$ and $w_{pos}=1.0$. Fig. 4 visualizes optimization procedure.
\begin{figure}
    \includegraphics[width=\columnwidth]{figures/3_method_optimization.png}
    \caption{Optimization procedure of our differentiable rendering. From generated target noise-free geometric clue $G_\mathcal{C}$ and input noisy geometric clue $G_{lw}$, we minimize $L_2$ distance between two clues. Note that we additionally penalize aggressive vertex evolve, however it is skipped in the figure.}
    \label{fig:optimization}
\end{figure}
\section{Results}
We have implemented our method on top of general differentiable renderer\cite{ravi2020accelerating}. 
For all tests, we used Intel Xeon E5-2687W CPU machine with 3.0GHz, and GeForce RTX 2080 TI for CUDA-optimized differentiable renderer. 
We optimized our mesh with Stochastic Gradient Descent method with learning rate = 1.0, and momentum = 0.9. 
We placed one virtual light source at the camera center for all experiments, although this method is up to a number of lights and its positions.

\begin{figure*}
    \centering
    \includegraphics[width=\textwidth]{figures/4_result_comparison_with_previous_method.png}
    \caption{Comparison of result using previous method and our method. For target silhouette, we generated it from input mesh by render mesh with silhouette renderer, and it is enough to use since silhouette image itself does not hold any depth information other than boundary information of target mesh. \textbf{Previous} result fails to converge to perceptive geometric shape hidden in target images, whereas \textbf{Proposed} result successfully reduces noise. Note that \textbf{Proposed} method only observes $\mathbf{\mathcal{C}}$ in order to infer noise-free geometry, whereas Previous method requires both $\mathbf{\mathcal{C}}$ and $\mathbf{S}$.}
    \label{fig:comparison_with_previous_method}
\end{figure*}

\subsection{Comparison with Previous Work}
We compare our method with standard differentiable rendering approach used in differentiable renderer\cite{liu2019soft}. 
We slightly modified its workflow in order to evaluate performance under identical conditions (please refer \ref{fig:difference_simple_mesh_and_tsdf_mesh} for details). 
Specifically, we limit amount of target view to 1, as it is impossible to synthetically generate clean target using existing SLAM sequences. 
For silhouette image, we used silhouette of input TSDF mesh as silhouette itself does not take into account detailed depth information as they represent boundary of mesh \PJ{TODO: to footnote}(Hence, previous methods required silhouette images from different camera extrinsic in order to optimize the shape of input mesh). 
We follow losses and its weights as suggested in previous method\cite{ravi2020accelerating}, which is defined as: 
\begin{multline*}
    \mathcal{L}_{prev}=w_{sil}\cdot L_{sil}+w_{rgb}\cdot L_{rgb}+w_{edge}\cdot L_{edge}+ \\ 
    w_{normal}\cdot L_{normal}+w_{laplacian}\cdot L_{laplacian}, 
\end{multline*}
where $L_{sil}$, $L_{rgb}$ is silhouette, and color difference between input and target, respectively. 
$L_{edge}$, $L_{normal}$, and $L_{laplacian}$ stands for mesh edge length consistency, mesh normal orientation consistency, and mesh Laplacian loss, respectively. 
Note that mesh losses $L_{edge}$, $L_{normal}$, $L_{laplacian}$ is developed from traditional geometry processing field, thus they do not reflect geometric features that are shown in target images. 
For brevity, we skip detailed equations for each loss. 
We used $w_{sil}=1.0$, $w_{rgb}=1.0$, $w_{edge}=1.0$, $w_{normal}=0.01$, $w_{laplacian}=1.0$ as authors suggested.
Fig. 5 shows optimized mesh with loss from previous work. 
Resulting mesh preserves its silhouette information, as they provided silhouette loss to prevent vertices to evolve out of its silhouettes. 
However, vertices within silhouette images are evolved without any shape constraints supervised by target images. 
Those vertices are guided to satisfy input mesh’s internal properties, as mesh losses does not consider geometric clues in target images. 
This results in optimized mesh failed to optimize, while only preserving boundary.

\subsection{Validation of Proposed Method}
In Figure 6, we plotted our loss graph to ensure that our loss function forms a convex function. 
Our proposed lightweight loss is monotonically decreased as the optimization continues. 
Note that our positional loss is not minimized, as we want positional loss to act similar with regularization term, which prevents evolving vertices getting not to far from its original position. 
Nevertheless, the total loss forms a convex function as the actual contribution of positional loss is small enough than lightweight loss. \PJ{TODO: vaildate with other scenes.}

\begin{figure}
    \includegraphics[width=\columnwidth]{figures/4_result_loss_plot_with_images.png}
    \caption{Loss plot with loss images at specific iterations (here, 0, 150, and 300 is used). Our lightweight loss successfully minimizes vertex noise by observing target noise-free geometric information $G_\mathcal{C}$. Note that $L_{pos}$ never converges to zero, as they act as constraint to vertices not to evolve too far from its original position, thus have non-zero values among all iterations.}
    \label{fig:loss_plot_with_images}
\end{figure}

\subsection{Limitations \& Future Directions}
We summarize future directions of our method, including failure case and promising extension of proposed method.

\paragraph{Adding Robust Positional Constraint Loss}
Although our method can drastically reduce mesh noise, we found that optimized mesh is not maintaining topologically correct structure. 
Specifically, we observed self-intersections between neighboring faces. 
This problem can be naturally arisen since we did not consider relationship between neighboring vertices in our final loss. 
Therefore, though vertices are evolved to have reduced noise, their evolving direction is totally random, lead to occur self-intersections between neighboring faces by corresponding (incorrectly evolved) neighboring vertices. 
We illustrated this failure case in Figure 7.


\begin{figure}
    \includegraphics[width=\columnwidth]{figures/4_future_edge_fitting.png}
    \caption{Inset of each gradient. $G_{lw}$ is corrupted since geometric normal deviates around edge.}
    \label{fig:edge_fitting}
\end{figure}


\begin{figure*}
    \includegraphics[width=\textwidth]{figures/4_future_work_self_intersection.png}
    \caption{Comparison of input mesh and optimized mesh. We zoomed up each geometry corresponds to image inset. 
    \textbf{Black artifacts} are observed in zoomed view in optimized mesh. We found that those are self-intersection between neighboring faces, since our positional constraint $L_{pos}$ does not consider distances between neighboring vertices. 
    This phenomenon is illustrated at the rightmost images of each zoomed view. 
    (a) For each iteration, vertices are optimized while only considering flatness of surfaces, not for topological correctness of geometry. 
    Star, circle, and arrow indicates light position, vertices, and gradient (evolving direction after current optimization step), respectively. 
    Here, \textbf{\textcolor{ForestGreen}{Green vertex}} is optimized to \textbf{\textcolor{ForestGreen}{Green arrow}} direction so that it occludes another face. 
    (b) Consequently, there exists a self-intersection between neighboring faces (\textbf{Black line}) although the entire mesh noise is reduced, resulting self-occluded artifact.}
    \label{fig:self_intersection}
\end{figure*}

\paragraph{Edge Fitting}
Although our method is able to minimize noisy vertices by treating $G_\mathcal{C}$ as noise-free geometric clue, however it failed to clean out noisy vertices around geometric edges. 
We found that there is gradient magnitude difference between two images, as on edges $G_{lw}$ tends to have large deviation since it is the place where geometric normal hugely differs. 
Our current $L_2$ loss of gradients cannot reflect this case, as they are evaluated pixel-by-pixel. 
We visualized this case in Figure 8. 
Weighting color gradients by using gradients from depth, say $G_\mathcal{D}$, or developing an erosion kernel for $G_\mathcal{C}$ to reliably cover $G_{lw}$ seems worth trying.


\section{Conclusion}
In this report, we proposed a method that optimizes input TSDF mesh generated from single RGB-D pair, by exploiting noise-free geometric clues hidden in color image. 
Our method differs from most of previous methods as they required to optimize shape from multiple takes of target scene, which is hard to directly apply to SLAM dataset. 
We show that our method outperforms previous differentiable rendering setup. 
Future work on improving robustness in terms of optimizing edgy geometries, preventing topological correctness of input mesh would be beneficial to solve problem that is defined as ill-posed in SLAM.


\end{document}
